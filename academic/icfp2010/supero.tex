\documentclass[draft]{sigplanconf}

\usepackage{amsmath}
\usepackage{amssymb}
\usepackage{natbib}
\usepackage{multirow}
\usepackage{setspace}
\usepackage{balance}

\include{paper}
%include paper.fmt

% Consistency:
% v,w,x,y,xs,ys,z,zs are all variables
% e is an expression
% p is a pattern
% f,g are functions
% m,n are lengths
% i,j are indexes

%format <? = "[\!["
%format ?> = "]\!]"
%format += = "+\!\!\!\!="
%format ==> = "\Longrightarrow{}"
%format <=| = "\unlhd{}"
%format <| = "\lhd{}"

%format w_1  = "\Varid{w_1}"
%format w_2  = "\Varid{w_2}"
%format w_3  = "\Varid{w_3}"
%format w_12  = "\Varid{w_12}"
%format w_123  = "\Varid{w_123}"
%format v_3  = "\Varid{v_3}"
%format v_4  = "\Varid{v_4}"
%format map_1 = "\Varid{map_1}"
%format map_2 = "\Varid{map_2}"
%format e_1' = "\Varid{e_1^{\prime}}"
%format e_2' = "\Varid{e_2^{\prime}}"
%format e_m' = "\Varid{e_m^{\prime}}"
%format e_1 = "\Varid{e_1}"
%format p_m = "\Varid{p_m}"
%format s_m = "\Varid{s_m}"
%format v_j = "\Varid{v_j}"
%format v_5 = "\Varid{v_5}"
%format v_6 = "\Varid{v_6}"
%format x_3 = "\Varid{x_3}"
%format x_4 = "\Varid{x_4}"
%format x_m = "\Varid{x_m}"
%format s_m_1 = "\Varid{s_{m-1}}"
%format w_m = "\Varid{w_m}"

\newcommand{\unknown}{XXX}
\newcommand{\name}[3]{\ensuremath{\langle\mathsf{#1},\mathsf{#2},\mathsf{#3}\rangle}}
\newcommand{\lemma}[1]{\subsubsection*{\textit{Lemma: #1}}}

\newcommand{\setsup}{\supset_{\mathrm{set}}}
\newcommand{\setequiv}{\equiv_{\mathrm{set}}}
\newcommand{\bagsub}{\subset_{\mathrm{bag}}}
\newcommand{\bagequiv}{\equiv_{\mathrm{bag}}}


\begin{document}

\conferenceinfo{ICFP 2010}{}
% \CopyrightYear{2009}
% \copyrightdata{978-1-60558-508-6/09/09}

\titlebanner{\today{} - \currenttime{}}        % These are ignored unless
\preprintfooter{}   % 'preprint' option specified.

\title{Simpler Supercompilation}
% \subtitle{}

\authorinfo{Neil Mitchell}
           {Standard Chartered, UK}
           {\verb"ndmitchell@gmail.com"}

\maketitle

\begin{abstract}
Supercompilation is a program optimisation technique, which can often eliminate the overhead of programmer abstractions. We present a new design for a supercompiler, rethinking many of the decisions that have become common in previous supercompilation work. The result is a supercompiler that we find simpler, and offers good performance both at compile time and runtime. We have implemented our supercompiler and benchmarked it on a selection of programs from the nofib benchmark suite, which run faster than both GHC and our previous supercompiler. We then put our supercompiler to practical use, optimising an HTML parsing library, and checking the equivalence of open expressions used as rewrite rules.
\end{abstract}

\category{D.3}{Software}{Programming Languages}

\terms
Languages

\keywords
Haskell, optimisation, supercompilation

\section{Introduction}

\todo{perhaps switch to tagsoup as the openning example, so I can come back to it?}

Supercompilation is a technique for program optimisation, that is particularly suited to removing the overhead introduced by abstractions. Consider a program that counts the number of words read from the standard input -- in Haskell \cite{haskell} this can be compactly written as:

\begin{code}
main = print . length . words =<< getContents
\end{code}

Reading the program right to left, we first read the standard input as a string (|getContents|), then split it in to words (|words|), count the number of words (|length|), and print the result (|print|). An equivalent C program is unlikely to use such a high degree of abstraction, and is more likely to get characters and operate on them in a loop using some state which is updated.

Sadly the C program is three times faster, even using the advanced optimising compiler GHC \cite{GHC}. The abstractions that make the program concise have a significant runtime cost. In a previous paper on supercompilation \cite{me:supero} we showed how supercompilation can remove these abstractions, to the stage where the Haskell is even faster than the C version (by about 6\%). In the Haskell program after optimisation all the intermediate lists have been removed, and the |length . words| part of the pipeline is translated into a state machine.

One informal description of supercompilation is that you simply ``run the program at compile time''. This description leads to two questions -- what happens if you are blocked on information only available at runtime, and how do you ensure termination? Answering these questions provides the design for a supercompiler. This paper strives to provide simple answers, but which do not reduce the optimisation power. In doing so, we make a large number of departures from the current concensus on supercompiler design.

We hope that from the descriptions given in this paper a user is able to write their own supercompiler. Indeed, we have written a supercompiler following this design which is available online \unknown{}.

\subsection{Contributions}

Our primary contribution is the design of a new supercompiler (\S\ref{sec:method}). Our supercompiler has many differences from previous supercompilers (\S\ref{sec:comparison}), including a new core language, a substantially different treatment of let expressions, a transformation that never manipulates inside expressions and an entirely new termination criteria. The result is simpler than previous supercompilers, yet still powerful.

In addition, we make the following contributions:

\begin{itemize}
\item We give examples of how our supercompiler performs (\S\ref{sec:examples}), including how it subsumes list fusion and specialisation, and what happens when the termination criteria are needed.
\item We benchmark the supercompiler on a small range of examples (\S\ref{sec:benchmarks}) -- our supercompiler achieves an improvement of \unknown{}\% when used in combination with GHC, as opposed to GHC alone.
\item We use our supercompiler to optimise an HTML parser, showing a \unknown{}\% speed increase (\S\ref{sec:tagsoup}). This HTML parser is run daily on 40Gb of input, and previously consumed over two hours of CPU time per day.
\item We use our supercompiler to prove the equality of open expressions for a program rewriter (\S\ref{sec:hlint}). As a result we found one bug.
\end{itemize}

\section{Method}
\label{sec:method}

This section describes our supercompiler. We first present a Core language (\S\ref{sec:core}), along with simplification rules (\S\ref{sec:simplify}). We then present the overall algorithm (\S\ref{sec:manager}), which uses answers to the following questions:

\todo{reduce the space between bullet points}
\begin{itemize}
\item How do you evaluate an open term? (\S\ref{sec:eval})
\item What happens if you can't evaluate an open term further? (\S\ref{sec:eval_split})
\item How do you know when to stop? (\S\ref{sec:term})
\item What happens if you have to stop? (\S\ref{sec:term_split})
\end{itemize}

\noindent Throughout this section we use the following example:

\begin{code}
root g f x = map g (map f x)

map f x = case  x of
                []    -> []
                y:ys  -> f y : map f ys
\end{code}

This example applies |map| twice -- the expression |map f x| produces a list that is immediately consumed by |map g|. A good supercompiler should remove the intermediate list.

\subsection{Core Language}
\label{sec:core}

\begin{figure}
\begin{code}
type Var   =   String -- variable names
type Fun   =   String -- function names
type Con   =   String -- constructor names

data Exp   =   Var Var
           |   Fun Fun
           |   Con Con [Var]
           |   App Var Var
           |   Let [(Var,Exp)] Var
           |   Case Var [(Pat, Exp)]

type Pat   =   Exp -- restricted to |Con|
\end{code}
\caption{Core Language}
\label{fig:core}
\end{figure}

Our Core language for expressions is given in Figure \ref{fig:core}, and has much in common with Administrative Normal Form \cite{flanagan:continuations}. There are a few points to note:

\begin{itemize}
\item We require variables in many places that would normally permit expressions. A standard Core language (such as from \citet{ghc_core}) could be translated to ours by inserting let bindings.
\item In many Core languages a distinction is made between standard let bindings and recursive let bindings. We allow expressions bound at a let to refer to variables bound at the same let, but disallow recursion. An alternative description is that we allow recursive let, but only if there is an equivalent nested non-recursive let that can represent the same expression. For example we allow |let x = y; y = C in C| but not |let x = y; y = x in C|.
\item We don't have default patterns in case expressions, or literals. Both can be added (see \S\ref{sec:extensions}), but are of little interest when describing a supercompiler.
\item We assume that the Core language is well-typed, in particular that we never over-apply a constructor.
\end{itemize}

We write expressions using standard Haskell syntax (i.e. |let| for |Let|, |case| for |Case| etc.). Rewriting the |map/map| example in our Core language gives:

\begin{code}
root g f x =  let  v_1 =  map
                   v_2 =  let  w_1 = map
                               w_2 = w_1 f
                               w_3 = w_2 x
                          in   w_3
                   v_3 =  v_1 g
                   v_4 =  v_3 v_2
              in   v_4

map f x = case  x of
                []    ->   let  q    = []
                           in   q
                y:ys  ->   let  v_1  = f y
                                w_1  = map
                                w_2  = w_1 f
                                w_3  = w_2 ys
                                q    = (:) v_1 w_3
                           in   q
\end{code}

Our Core language is rather verbose, so sometimes we write expressions using a standard Core language, assuming these are translated to our Core language when necessary. For example, we might write |map| as:

\begin{code}
map f x = case  x of
                []    -> []
                y:ys  -> f y : map f ys
\end{code}

\subsection{Simplified Core}
\label{sec:simplify}

We now define a simplified form for our Core language. When working with Core expressions we assume they are always simplified, and after constructing new expressions we always simplify them. Our simplified form requires that all variables are unique, that the root of a function be a let expression, and that the expressions bound at the root let must \textit{not} have the following form:

\begin{itemize}
\item |App v w|, where |v| is bound to a |Con| -- the |App| can be replaced with a Con of higher arity.
\item |Case v w|, where |v| is bound to a |Con| -- the |Case| can be replaced with the appropriate alternative.
\item |Let vs w| -- the nested bindings can be brought to the root |Let|, renaming variables if necessary.
\item |Var v| -- we can remove the binding by replacing the bound variable with |v| everywhere.
\end{itemize}

We can also remove any bindings which are not used. These simplifications may result in a root let expression with no bindings. As an example, we can apply these rules to the following expression:

\begin{code}
let  v_1  = f
     v_2  = Con x
     v_3  = v_2 y
     v_4  = let w_1 = y in v_1 w_1
     v_5  = case v_3 of Con a b -> v_4 a
in   v_5
\end{code}

\noindent To give:

\begin{code}
let  v_1  = f
     v_4  = v_1 y
     v_5  = v_4 x
in   v_5
\end{code}

As another example, the simplified version of |root| is:

\begin{code}
root g f x =  let  v_1  = map
                   w_1  = map
                   w_2  = w_1 f
                   w_3  = w_2 x
                   v_3  = v_1 g
                   v_4  = v_3 w_3
              in   v_4
\end{code}

\subsection{Manager}
\label{sec:manager}

\begin{figure}
\begin{code}
data Lambda = Lambda [Var] Exp
type Env = Fun -> Lambda
data Tree = Tree  {pre :: Lambda
                  ,gen :: [Fun] -> Lambda
                  , children :: [Tree]}

supercompile :: Env -> [(Fun,Lambda)]
supercompile env =
    assign $ flatten $ optimise env $ env "main"

optimise :: Env -> Lambda -> Tree
optimise env = f []
    where  f t x | terminate (<=|) t x = g (stop t x) t
                 | otherwise = g (reduce env x) (x:t)
           g (gen,cs) t = Tree x gen (map (f t) cs)

reduce :: Env -> Lambda -> ([Fun] -> Lambda, [Lambda])
reduce env = f []
    where f t x | terminate (<|) t x = stop t x
                | Just x' <- step env x = f (x:t) x'
                | otherwise = split x

flatten :: Tree -> [Tree]
flatten = nubBy (\x y -> pre x == pre y) . f []
    where f seen t  =  if pre t `elem` seen then [] else
                       t : concatMap (f (t:seen)) (children t)

assign :: [Tree] -> [(Fun,Lambda)]
assign ts = [(f t, gen t (map f (children t))) |  t <- ts]
    where f = flip lookup . zip freshNames (map pre ts)
\end{code}
\caption{The |supercompile| function.}
\label{fig:manager}
\end{figure}

\begin{figure}
\begin{code}
step :: Env -> Lambda -> Maybe Lambda -- \S\ref{sec:eval}
split :: Lambda -> ([Fun] -> Lambda, [Lambda]) -- \S\ref{sec:eval_split}

type History = [Lambda]
(<|),(<=) :: Lambda -> Lambda -> Bool -- \S\ref{sec:term}
terminate  :: (Lambda -> Lambda -> Bool)
           -> History -> Lambda -> Bool -- \S\ref{sec:term}
stop :: History -> Lambda -> ([Fun] -> Lambda, [Lambda])
    -- \S\ref{sec:term_split}
\end{code}
\caption{Auxiliary definitions for Figure \ref{fig:manager}.}
\label{fig:manager2}
\end{figure}

Our supercompiler is based around a manager, that integrates the answers to the questions of supercompilation. The manager itself has two main purposes: to ensure termination, and to tie back recursive functions. In our experience the act of tieing recursion is often one of the trickiest aspects when writing a supercompiler, so we deliberately choose to give plenty of detail. The code for our manager is given in Figure \ref{fig:manager}, making use of a some auxiliary functions whose types are given in \ref{fig:manager2}. We start by giving an intuition for how the manager works, then describe each part separately.

The supercompiler takes a source program, and generates a target program. Functions in these programs are distinct -- target expressions cannot refer to source functions. The source and target program are semantically identical, but it is hoped the target program runs faster. We use the type |Lambda| to represent an expression enclosed within a lambda, and the type |Env| to represent a mapping from function names to lambda expressions. We can think of |Lambda| as a function, or as an \textit{open expression} whose free variables are listed in the lambda.

The manager first builds a tree tree, where each node in the tree is has a source expression (|pre|). The associated target expression may call other target functions, but these functions do not yet have names. Therefore, we store target expressions as a generator that when given the function names produces the target expression (|gen|), and a list of trees representing the target functions it needs to call (|children|). We then flatten this tree, ensuring identical functions are only represented once, and supply names to each node to generate the target program. If a target function is recursive then the initial tree will be infinite, but the flattened tree will always be finite due to the termination scheme defined in \S\ref{sec:term}.

\newcommand{\function}[1]{\paragraph{\textsf{#1:}}\hspace{-3mm}}

\function{supercompile} This function puts all the parts together. Reading from right to left, we first generate a potentially infinite tree by optimising the expression |main|, we then flatten the tree to a finite number of resultant functions, and finally assign names to each of the result functions.

\function{optimise} This function constructs the tree of result functions. While the tree may be infinite, we demand that any infinite path from the root must encounter the same |pre| value more than once. We require that for any infinite sequence |ts| with no repeats, there must exist an |i| such that |terminate (<=||) (take i ts) (ts !! i+1)| returns |True|. If we are forced to terminate we call |stop|, which splits the expression into several subexpressions. We require that |split ts| only produces subexpressions which pass the termination test. If the termination criteria do not force us to stop, then we call |reduce| to evaluate the expression.

\function{reduce} This function optimises a function by repeatedly evaluating it by calling |step|. If we can't evaluate any further we call |split|. We use a local termination test to ensure the evaluation terminates. We require that for any infinite sequence |ts|, there must exist an |i| such that |terminate (<||) (take i ts) (ts !! i+1)| returns |True|.

\function{flatten} This function takes a tree and extracts a finite number of functions from it, assuming the termination restrictions given in |optimise|. Our |flatten| function will only keep one tree associated with each source expression. These trees may be different if one resulted from a call to |stop|, while another from a call to |reduce| -- but all are semantically equivalent.

\function{assign} This function assigns names to each target function, and constructs the target expressions by calling |gen|. We assume the function |freshNames| returns the infinite list of fresh function names.

\subsubsection{The Example}
\label{sec:manager_example}

Revisiting our initial example, |supercompile| first calls |optimise| with:

\begin{code}
\g f x -> map g (map f x)
\end{code}

The termination context is empty, so we call |reduce| which calls |step| repeatedly until we reach the expression:

\begin{code}
\g f x ->  let  v = case  w of
                          [] -> []
                          y:ys -> g y : map g ys
                w = case  x of
                          [] -> []
                          z:zs -> f z : map f zs
           in   v
\end{code}

The |step| function now returns |Nothing|, since the root cannot be reduced without the result of |x|. We therefore call |split| which results in:

\begin{code}
\g f x -> case  x of
                []    -> <? let v = ...; w = ...; x = [] in v ?>
                z:zs  -> <? let v = ...; w = ...; x = z:zs in v ?>
\end{code}

To represent a value of type |([Fun] -> Lambda, [Lambda])| we use the |<? bullet ?>| notation to identify particular children of an expression. The first component of the resulting pair takes a function name for each child and constructs the target expression. The second component is the list of children. For each child, all the free variables needed become arguments to the lambda, and are passed forward by the target expression.

When optimising the first child expression, where |x = []|, the simplification rules from \S\ref{sec:simplify} immediately produce |[]| as the result. For the other child, we first get:

\begin{code}
\g f z zs ->
    let  v = case  w  of [] -> []; y  :ys  -> g  y  : map g  ys
         w = case  x  of [] -> []; z  :zs  -> f  z  : map f  zs
         x = z:zs
    in   v
\end{code}

Which simplifies to:

\begin{code}
\g f z zs ->  let  v = q : qs
                   q = g y
                   qs = map g ys
                   y = f z
                   ys = map f zs
              in   v
\end{code}

Calling |step| produces |Nothing|, as the root of this expression is a |(:)| that can't be evaluated. We therefore call |split| which results in:

\begin{code}
\g f z zs ->  let  q = <? g (f z) ?>
                   qs = <? map g (map f zs) ?>
                   v = q : qs
              in   v
\end{code}

When optimising |g (f z)| we get no optimisation, as there is no available information. To optimise |map g (map f zs)| we do exactly the same steps as we have already done. However, the |flatten| function will spot that both nodes have the same |pre| expression (modulo free variables), and give them both the same name, creating a recursive function. We then assign names using |assign|. For the purposes of display (not optimisation), we apply a number of simplifications given in \S\ref{sec:postprocess}. The end result is:

\begin{code}
main g f x = case  x of
                   []    -> []
                   z:zs  -> g (f z) : main g f zs
\end{code}

The final version has automatically removed the intermediate list, with no extra knowledge about the |map| function or it's fusion rules.

\subsection{Evaluation}
\label{sec:eval}

\begin{figure}
\begin{code}
force :: Exp -> Maybe Var
force (Case  v _  )  = Just v
force (App   v _  )  = Just v
force (Var   v    )  = Just v
force _              = Nothing

stack :: Lambda -> [(Var, Exp)]
stack (Lambda _ (Let bind v)) = f v
    where f v = case  lookup v bind of
                      Nothing  -> []
                      Just x   -> maybe [] f (force x) ++ [(v,x)]
\end{code}
\caption{Function to compute the evaluation stack.}
\label{fig:stack}
\end{figure}

Evaluation is based around the |step| function. Given an expression, |step| inlines a function body and returns |Just|, or if no suitable function can be inlined returns |Nothing|. We always inline the function which would be evaluated next during normal evaluation, replacing the function with it's body, provided it has enough arguments not to result in a lambda.

To be more precise, we define the evaluation stack of an expression as the sequence of let bindings that would be evaluated in order, along with their associated expressions. A function to calculate the stack is given in Figure \ref{fig:stack}. Looking at the original example:

\begin{code}
\g f x =  let  v_1  = map
               w_1  = map
               w_2  = w_1 f
               w_3  = w_2 x
               v_3  = v_1 g
               v_4  = v_3 w_3
          in   v_4
\end{code}

The evaluation stack for this expression will be |[v_1,v_3,v_4]|. The stack is computed from the right, starting at the body of the let. From |v_4| we arrive at the expression |v_3 w_3|, which will need to first evaluate |v_3|. Similarly the evaluation of |v_3| will require evaluating |v_1| first. Therefore, to evaluate this expression we will start by evaluating |v_1|, and thus |map|.

We scan down the stack, find sufficient arguments, inline the body, and replace the expression bound to |v_4| with:

\begin{code}
let  f  = g
     x  = w_3
in   case  x of
           []    -> []
           y:ys  -> f y : map f ys
\end{code}

Simplification will immediately eliminate the |f| and |x| bindings.

More generally, we can match any expression with the following pattern:

\begin{code}
\free ->  let  s_1  = f
               s_2  = s_1 w_2
               s_m  = s_m_1 w_m
               v_1  = e_1
               v_n  = e_n
          in   v
where f = \x_2 x_m -> e'
\end{code}

Here we have used |s_1| to represent the first element of the stack, which must be a function. The elements |s_2..s_m| must all be applications on the stack, and must be equal to the arity of the lambda expression bound to |f|. We allow any other variables |v_1..v_n| bound to expressions |e_1..e_n| to be present. Given this configuration we can rewrite to:

\begin{code}
\free ->  let  s_1  = f
               s_2  = s_1 w_2
               s_m  = let x_2 = w_2; x_m = w_m in e'
               v_1  = e_1
               v_n  = e_n
          in   v
\end{code}

As always, after generating a new expression we immediately apply the simplification rules (\S\ref{sec:simplify}), which will eliminate the nested let expression bound to |s_m|.

\subsubsection{Special Empty Case}

As a special case, we replace the pattern:

\begin{code}
\free ->  let  v = f
          in   v
where f = \x_1 x_n -> e
\end{code}

With:

\begin{code}
\free x_1 x_n -> e
\end{code}

We include this rule to ensure that we can evaluate even the smallest expression, so even with a very restrictive termination function we can still make progress.

\subsection{Evaluation Splitting}
\label{sec:eval_split}

If evaluation cannot proceed we split to produce a target expression, and a list of child expressions for further optimisation, using the |<? bullet ?>| notation described in \S\ref{sec:manager_example}. When splitting an expression there are three concerns:

\paragraph{Permit further optimisation:} We aim to place any constructs blocking evaluation in the target expression, so that the children are suitable for further optimisation.

\paragraph{No loss of sharing:} A variable may not be duplicated if that causes it to be evaluated multiple times at runtime. The target program cannot remove sharing present in the source program.

\paragraph{Keep expressions together:} If we turn a bound variable in to a free variable we loose optimisation opportunities, as the right-hand side of the variable is no longer available for optimisation. We aim to keep as many expressions together as possible, but not at the cost of loosing sharing.

\smallskip
We split in one of three different ways, depending on the type of expression at the top of the stack (as described in \S\ref{sec:eval}). We now describe each of the three ways to split, in each case we start with an example, then define the general rule.

\subsubsection{Case Expression}

If the top of the stack is a case expression then the target is a similar case expression, and under each alternative we create a child expression with the case scrutinee bound to the appropriate pattern. For example, given:

\begin{code}
\x ->  let   v = case  x of
                       []    -> []
                       y:ys  -> add y ys
       in    v
\end{code}

We produce the residual expression:

\begin{code}
\x ->  case x of
       []    -> <?  let  v =  case x of [] -> []; y:ys -> add y ys
                         x =  []
                    in   x ?>
       y:ys  -> <?  let  v =  case x of [] -> []; y:ys -> add y ys
                         x =  y:ys
                    in   x ?>
\end{code}

Looking more closely at the second alternative, we start with the expression:

\begin{code}
\y ys ->  let  v  = case x of [] -> []; y:ys -> add y ys
               x  = y : ys
          in   v
\end{code}

This expression immediately simplifies to:

\begin{code}
\y ys ->  let  v = add y ys
          in   v
\end{code}

One important point is that for the first alternative we \textit{do not} pass in the variables |y| and |ys| -- these variables are not included in the lambda. In general we do not pass onwards any free variables which are obviously redundant after the simplification rules have been applied. By making this restriction we ensure that the number of free variables in a function cannot grow without bound -- as the expressions are bounded and therefore they can only use a finite number of free variables. We limit the free variables for all types of split operation.

The general rule is that the target is the case on the top of the stack, and the alternatives are the entire expression but with the scrutinee variable bound to the associated patterns. More generally, if |s_1| is the top of the stack:

\begin{code}
\free ->  let  s_1  = case x of p_1 -> e_1' ; p_m -> e_m'
               v_1  = e_1
               v_n  = e_n
          in   v
\end{code}

\noindent becomes:

\begin{code}
\free -> case x of
    p_1  -> <? let  s_1 = case x of p_1 -> e_1'; p_m -> e_m'
                    v_1 = e_1; v_n = e_n; x = p_1 in v ?>
    p_m  -> <? let  s_1 = case x of p_1 -> e_1'; p_m -> e_m'
                    v_1 = e_1; v_n = e_n; x = p_m in v ?>
\end{code}

\subsubsection{Function}
\label{sec:eval_split_function}

If the top of the stack is a function when we get to split, then there must be insufficient arguments for evaluation to take place. To deal with the lack of arguments we introduce a lambda. The key point when introducing a lambda is that we do not reduce sharing. Consider the following example:

\begin{code}
\x ->  let  s_1 = add
            s_2 = s_1 v_2
            v_1 = expensive
            v_2 = v_1 x
       in   s_2
\end{code}

Here the |add| function takes two arguments, but has only been given one. It is tempting to rewrite |\x -> ...| as |\x y -> ...|, but this potentially duplicates the expensive computation of |v_2|. Instead we take the variables on the stack (|s_1| and |s_2|) and duplicate their bindings everywhere:

\begin{code}
\x ->  let  v_1 =  let s_1 = add; s_2 = s_1 v_2 in expensive
            v_2 =  let s_1 = add; s_2 = s_1 v_2 in v_1 x
       in   \y ->  let s_1 = add; s_2 = s_1 v_2 in s_2 y
\end{code}

We immediately discard any stack variables which introduce recursion (i.e. |s_2| bound under |v_2|), then simplify all the bindings:

\begin{code}
\x ->  let  v_1 =  expensive
            v_2 =  v_1 x
       in   \y ->  let s_1 = add; s_2 = s_1 v_2 in s_2 y
\end{code}

We could now produce the residual expression:

\begin{code}
\x ->  let  v_1 =  <? expensive ?>
            v_2 =  <? v_1 x ?>
       in   \y ->  <? let s_1 = add; s_2 = s_1 v_2 in s_2 y ?>
\end{code}

However, we have now split the bindings for |v_1| and |v_2| apart, when there is no real need. We therefore move binding |v_1| under |v_2|, because it is only referred to by |v_2|, to give:

\begin{code}
\x ->  let  v_2 =  <? let v_1 = expensive in v_1 x ?>
       in   \y ->  <? let s_1 = add; s_2 = s_1 v_2 in y ?>
\end{code}

We will now continue to optimise the body of |v_2| and the under the lambda introducing |y|. Provided |add| has arity two, we'll now be able to inline |add|. We have duplicated the partial applications along the stack, but that is acceptable as they are cheap and have a small fixed cost. To accommodate the inner lambda in the target expression we can either permit target expressions to be a richer language than our Core language from \S\ref{sec:core}, or we can arrange for the final argument of the child lambda to be |y|, in which case it can be passed by partial application rather than explicit lambda.

More generally, given:

\begin{code}
\free ->  let  v_1  = e_1
               v_n  = e_n
               s_1  = f
               s_2  = e_2'
               s_m  = e_m'
          in   s_m
\end{code}

We rewrite:

\begin{code}
\free ->  let  v_1 =  <? let s_1 = f; s_2 = e_2'; s_m = e_m' in e_1 ?>
               v_n =  <? let s_1 = f; s_2 = e_2'; s_m = e_m' in e_n ?>
          in   \y ->  <? let s_1 = f; s_2 = e_2'; s_m = e_m' in s_m y ?>
\end{code}

We then repeatedly move any binding |v_i| under |v_j| if |v_i| is only used within the body of |v_j|.

\subsubsection{Anything Else}

The final rule applies to any expression where the top of the stack is not a case expression or function, including a constructor, a variable, and an application to an unknown variable. Given the example:

\begin{code}
\x y ->  let  v_1 = expensive
              v_2 = v_1 x
              v_3 = Con v_2 y v_2
         in   v_3
\end{code}

We simply put all the variables other than the first on the stack inside |<? bullet ?>| brackets:

\begin{code}
\x y ->  let  v_1 = <? expensive ?>
              v_2 = <? v_1 x ?>
              v_3 = Con v_2 y v_2
         in   v_3
\end{code}

We then perform the same sharing transformation as for functions, noting that |v_1| is only used within |v_2|, to give:

\begin{code}
\x y ->  let  v_2 = <? let v_1 = expensive in v_1 x ?>
              v_3 = Con v_2 x v_2
         in   v_3
\end{code}

More generally, given an expression:

\begin{code}
\free ->  let  s_1 = e_1'
               v_1 = e_1
               v_n = e_n
          in   v
\end{code}

We rewrite to:

\begin{code}
\free ->  let  s_1 = e_1'
               v_1 = <? e_1 ?>
               v_n = <? e_n ?>
          in   v
\end{code}

We then repeatedly move any binding |v_i| under |v_j| if |v_i| is only used within the body of |v_j|. This rule is not suitable if |e_1'| is a function, as that expression ends up in the target program. This rule is suitable for case expressions, but is unnecessarily restrictive as case doesn't need to worry about sharing.

\subsection{Termination}
\label{sec:term}

The termination rule is responsible for ensuring that whenever we proceed along a list of expressions we eventually stop. The intuition is that each expression contains a list of expressions at the root let, and each expression tracks where it came from in the source program. Each successive root let expression must contain either new subexpressions, or fewer subexpressions compared to previous versions.

In this section we first describe the |terminate|, $\lhd$ and $\unlhd$ functions from a mathematical perspective, then how we apply these functions to expressions. Finally we show an example of how the termination rule catches non-termination.

\subsubsection{Termination Rule}

Our termination orderings are defined over bags (also known as multisets) of values drawn from a finite alphabet $\Sigma$. A bag of values is unordered, but may contain elements more than once. We define our two orderings as:

\[
x \lhd y = x \setsup y \vee x \bagsub y
\]
\[
x \unlhd y = x \lhd y \vee x \equiv y
\]

A sequence $x_1 \ldots x_n$ is well-formed under $\lhd$ if for all indices $i < j$, $x_j \lhd x_i$ (and respectively for $\unlhd$).

The following sequences are well-formed under both $\unlhd$ and $\lhd$:

\begin{code}
[a,aaaaab,aaab,b]
[abc,ab,ac,a]
[aaaaabbb,aaab,aab]
\end{code}

The following sequences are well-formed under $\unlhd$, but not under $\lhd$:

\begin{code}
[aaa,aaa]
[aabb,ab,abb]
\end{code}

The following sequences are not well-formed under $\unlhd$ or $\lhd$:

\begin{code}
[abc,acc]
[aa,aaa]
\end{code}

We define the |terminate| function from Figure \ref{fig:manager2} as:

\begin{code}
terminate  :: (Lambda -> Lambda -> Bool)
           -> History -> Lambda -> Bool
terminate (<) hist x = not $ all (x <) hist
\end{code}

The |terminate| function checks that given a well-formed sequence (|hist|), adding the expression |x| will keep the sequence well-formed.

\lemma{Any well-formed sequence under $\lhd$ is finite}

Given a finite alphabet $\Sigma$, any well-formed sequence under $\lhd$ is finite. Consider a well-formed sequence $x_1\ldots$. We can partition this sequence into at most $2^\Sigma$ subsequences using set equality. Consider any subsequence $y_1\ldots$. For any two elements in the subsequence, $y_i \setsup y_j$ will be false, because $y_i \setequiv y_j$ due to the partitioning. Therefore, for the sequence to be well-formed, $i < j \Rightarrow y_j \bagsub y_i$. To be a subbag means the cardinality of $y_j$ must be strictly lower than that of $y_i$. Therefore there can be at most $\#y_1+1$ elements in any particular subsequence. Combined with a finite number of subsequences, we show that any well-formed sequence is finite.

\lemma{Any well-formed sequence under $\unlhd$ has a finite number of distinct elements.}

The proof for $\unlhd$ builds on that of $\lhd$. For every subsequence, $i < j \Rightarrow y_j \bagsub y_i \vee y_j \bagequiv y_i$. As we progress along the subsequence the cardinality either decreases, or the element is identical. Therefore, there can be at most $\#y_1+1$ distinct elements. Combined with a finite number of subsequences, this gives a finite number of distinct elements in a well-formed sequence.

\subsubsection{Tracking Names}

Every expression in the source program is assigned a name. A name is a triple, \name{\mathit{f}}{\mathit{e}}{\mathit{c}} where $f$ is a function name, $e$ is an expression index and $c$ is a constructor count. We label every expression in the source program with $f$ being the function it comes from, $e$ being a unique index within that function and $c$ being $0$.\footnote{It is possible to merge the function name and subexpression index, by making the subexpression index globally unique, but this change would make it harder to understand and debug.} When manipulating expressions, we need to track and update names:

\begin{itemize}
\item If we rename bound variables, we do not update the expression names.
\item If we extract a subexpression we use the name already assigned to that subexpression.
\item If we insert a new constructor (when splitting on a case) we use the name assigned to the pattern in the associated case alternative.
\item If we add variables to the end of a constructor, we increase the constructor count. For example, |let v = C x; w = v y in ...| being transformed to |let v = C x; w = C x y in ...| would have the name for |C x y| set to the name of |C x| with the constructor count incremented.
\end{itemize}

We map an expression to a bag of names by taking the names of all subexpressions bound at the root let. We never look at the name of a root let expression, and do not update it when making transformations.

\lemma{For any source program, there are a finite number of names}

All subexpressions are assigned expression indices in advance, so there are only a finite number of function name/index values. We only increase the constructor count when increasing the number of arguments applied to a constructor, which is bounded by the arity of the constructor. Therefore, there are only a finite number of names.

\lemma{There are a finite number of expressions for any bag}

\todo{here}

Given a bag of names, there are only a finite number of expressions that could have generated it. We first assume that we always normalise the free variables in an expression by naming the let body |v_1|, and naming all other variables as they are reached from |v_1|. With this normalisation, a given number of subexpressions can only be combined in a finite number of ways. Since all subexpressions with the same name are equal modulo variable names, then the variable name normalisation makes them actually equal.

\lemma{The termination properties required by \S\ref{sec:manager} are satisfied}

Given a sequence with a finite number of bags, this translates to a sequence with a finite number of expressions. This property follows naturally if there are only a finite number of expressions for a given bag.


Combining these two lemmas, and the lemmas about $\lhd$ and $\unlhd$ we end up with an argument for the termination criteria required. To ensure the termination properties required by \S\ref{sec:manager} we require the following two properties:



The termination properties require a finite alphabet $\Sigma$, therefore we require a finite number of names. 


\subsubsection{Example}

Many small example programs do not make any real use of the termination criteria -- for example, the |map/map| example never hits the termination criteria. For a program to use the termination criteria it usually has to create a buffer which it uses later. The classic example of this pattern is |reverse|, which builds up an accumulated list which it uses later. The |reverse| style example results in termination being forced by the |optimise| function. We start with the program:

\begin{code}
main xs = rev [] xs
rev acc xs = case  xs of
                   []    -> acc
                   y:ys  -> rev (y:acc) ys
\end{code}

The |rev| function builds up an accumulator argument, which will be equal to the size of |xs|. If we tried to specialise the accumulator argument then we'd create an infinite number of specialisations. The |optimise| function starts with an empty termination context and the expression |rev [] xs|, and calls |reduce|, resulting in:

\begin{code}
\xs -> case  xs of
             []    -> <? [] ?>
             y:ys  -> <? rev (y:acc) ys ?>
\end{code}

Focusing on the second alternative, we now add |rev [] xs| to the termination context, and continue optimising |rev (y:acc) ys|. This leads to the sequence of expressions:

\begin{code}
\x_1 -> rev [] x_1
\x_1 x_2 -> rev (x_1:[]) x_2
\x_1 x_2 x_3 -> rev (x_1:x_2:[]) x_3
...
\end{code}

We can rewrite these expressions in our core language, with annotations for the names:

\newlength{\lenmain}
\newlength{\lenrev}
\settowidth{\lenmain}{|main|}
\settowidth{\lenrev}{|rev|}
\addtolength{\lenmain}{-\lenrev}
\newcommand{\namemain}[1]{\name{main}{#1}{0}}
\newcommand{\namerev}[1]{\name{rev\hspace{\lenmain}}{#1}{0}\hspace{1mm}}

\begin{code}
\x_1 ->
    let  v_1 = {-"\namemain{1}"-}  rev
         v_2 = {-"\namemain{2}"-}  []
         v_3 = {-"\namemain{3}"-}  v_1 v_2
         v_4 = {-"\namemain{4}"-}  v_3 x_1
    in   v_4
\x_1 x_2 ->
    let  v_1 = {-"\namerev{1}"-}   rev
         v_2 = {-"\namemain{2}"-}  []
         v_3 = {-"\namerev{2}"-}   x_1:v_2
         v_4 = {-"\namerev{3}"-}   v_1 v_3
         v_5 = {-"\namerev{4}"-}   v_4 x_2
    in   v_5
\x_1 x_2 x_3 ->
    let  v_1 = {-"\namerev{1}"-}   rev
         v_2 = {-"\namemain{2}"-}  []
         v_3 = {-"\namerev{2}"-}   x_1:v_2
         v_4 = {-"\namerev{2}"-}   x_2:v_3
         v_5 = {-"\namerev{3}"-}   v_1 v_4
         v_6 = {-"\namerev{4}"-}   v_5 x_3
    in   v_6
\end{code}

Applying our termination criteria, the first two expressions are a well-formed sequence, but comparing the second and third expressions we see that $e_2 \ntrianglelefteq e_3$. The first item is permitted because it is the first. The second is permitted as it introduces many new names (such as |{-"\name{rev}{1}{0}"-}|). The third item has no new names compared to the second, and is not a bag subset (it is in fact a bag superset). Therefore, when optimising, we call |stop| on the third expression.

We then end up with the sequence:

\begin{code}
\x_1 -> rev [] x_1                          -- |reduce|
\x_1 x_2 -> rev (x_1:[]) x_2                -- |reduce|
\x_1 x_2 x_3 -> rev (x_1:x_2:[]) x_3        -- |stop|
\x_1 x_2 x_3 -> rev (x_1:x_2) x_3           -- |reduce|
\x_1 x_2 x_3 x_4 -> rev (x_1:x_2:x_3) x_4   -- |stop|
\x_1 x_2 x_3 -> rev (x_1:x_2) x_3           -- |reduce|
\x_1 x_2 x_3 x_4 -> rev (x_1:x_2:x_3) x_4   -- |stop|
\x_1 x_2 x_3 -> rev (x_1:x_2) x_3           -- |reduce|
... -- repeat the last 2 lines
\end{code}

As required, we only have a finite number of unique expressions, and will end up with a recursive function in the target program.

\subsection{Termination Splitting}
\label{sec:term_split}

If we are forced to terminate we call |stop|, which splits the expression into several subexpressions, much like |split| from \S\ref{sec:eval_split}. We require that |split ts x| only produces subexpressions which pass the termination test |terminate (<=||) ts|. We trivially achieve this by using the termination criteria when writing |split|.

We first split everything in to a separate variable, i.e:

\begin{code}
\free ->  let  v_1 = <? e_1 ?>
               v_n = <? e_n ?>
          in   v
\end{code}

Now merge variable |v_i| under |v_j| provided:

\begin{itemize}
\item The function |terminate (<=||) t| does not kick in.
\item They don't loose sharing.
\end{itemize}

We don't need to ensure anything too bad, just that at the end we no longer hit the termination criteria for a finite nuber of times. If we were to bind all variables separately at the top-level, then we'd no longer have termination (by the rule of singleton sets never being too small) under |(<=||)|. And after the empty bag we never go anywhere.

\subsection{Argument for Correctness}

We have not shown the correctness of our supercompiler, but have tested it extensively. The termination argument follows from each section separately, in that we only ever transform an expression in semantics preserving ways.

\subsection{Post-processing}
\label{sec:postprocess}

We often need to post-process quite a bit, in particular we make the following simple transformations:

If two functions are equal we can make them the same.

If a function is called in exactly one place we can inline it.

We could have merged these functions in with the rest of the transformations but it would be a lot harder.

\subsection{Comparison to Other Supercompilers}
\label{sec:comparison}

It has no lambda, which makes it different from most, but exactly like Jason Reich's first one.

It has let everywhere, unlike most supercompilers which almost relegate let to a second class system.

It doesn't use homeomorphic embedding.

It never rewrites expressions, allowing for the tantilising opportunity to partially evaluate the supercompiler.

Our example in \S\ref{sec:term} would have stopped one step earlier with homeomorphic embedding. Notes: The first would be a homeomorphic embedding - we are in this case less severe than the homeo, but that's not the goal. Because of the simplify rules we end up with a fairly normal form, so it's pretty good that way.

We use a local termination pile, which means the optimisation of a program is not based on what has passed before. This hopefully leads to better predictability.

It's simple. The details provided in this paper are complete -- we have left nothing out.

Better separation with the manager and the other aspects.

Much easier handling of recursive functions.

For termination we need an ordering $\lhd$ such that for every infinite sequence $e_1,e_2 \ldots$ there exist indicies $i < j$ such that $e_i \ntriangleleft e_j$. Orderings which satisfy this criteria are known as well-quasi orderings, and come from Kruskal's Tree Theorem \cite{kruskal:tree}. Most other supercompilers rely on the homeomorphic embedding \cite{leuschel:homeomorphic} over expressions. We instead rely on an ordering over bags (also known as multisets) of values drawn from the finite alphabet $\Sigma$. Our ordering is given previously. Our ordering $\lhd$ is a well-quasi ordering, and $\unlhd$ is the same but with equality.

\subsubsection{No Lambda}
\label{sec:nolambda}

We don't have lambdas because to simplify them you can duplicate an arbitrary amount of crap. Given a lambda you can bind it twice with different values, which complicates things. Note that a case can only ever resolve to one value, and only bind variables once. While the addition of lambda would make the splitting when a function has insufficient free variables easier (\S\ref{sec:eval_split_function}), it makes it much harder to use our simplified termination ordering.

\subsection{Extensions}
\label{sec:extensions}

A real supercompiler needs slightly more flexibility, and the one we have implemented does indeed have all these

\subsubsection{Primitives}

Primitive functions are crucially important, but can be handled by simply treating them like free variables - everything works out.

\subsubsection{Literals}

We add primitives in to the language, and allow |Patt| to be both |Con| and |Lit|. In all other respects we can ignore primitives.

\subsubsection{Case Defaults}

Again, we just augment Patt. There are slight complexities to evaluation splitting when you are dealing with the default branch (don't bind the variable, just project out the right branch), but these aren't too challenging.

\subsubsection{Common Subexpression Elimination}

If we require a simplified let has no duplicate expressions we can do this trivially. We don't because we'd need to be careful about avoiding spaceleaks.

\subsubsection{Inlining Simple Functions}

One possible extension would be to automatically expand some functions whose termination was guaranteed - for example |($)|, |(.)|, |const|, |id|, |otherwise|. This could only be done for expressions which decreased in size.

\section{Examples}
\label{sec:examples}

Here are some examples of supercompiled functions, with a little bit of commentary.

\subsection{Specialisation}

Give map head as an example.

\subsection{Fusion}

You've already seen map/map, but it's important to remember that with our supercompiler lists are not special in any way. For example, if we wrote the identical program on the data type |data MyList alpha = MyNil || MyCons alpha (MyList alpha)| we'd get exactly the same fusion. To give an example:

\begin{code}
data Tree alpha = Leaf alpha | Branch (Tree alpha) (Tree alpha)

root x = flatten (mapTree (+1) x)
flatten -- with continuations
mapTree -- standard fmap on Tree
\end{code}

And we get the right result out.

\subsection{Residual values}

Of course, we can't specialise out |reverse|, for example:

\begin{code}
reverse = foldl (flip (:)) []
\end{code}

But we do get a nice efficient version of |reverse|, that has |foldl| specialised and optimised.

\section{Benchmarks}
\label{sec:benchmarks}

The standard nofib ones, eek!

\subsection{Execution Speed}

See how great we do, yay.

\subsection{Compilation Speed}

Our compiler is quite quick, we've broken in down in to the time to compile and the result. The tricks we use to speed up compilation are variable normalising, using a map for let expressions, and we could also reduce the termination keys in to |Int|, but we haven't bothered yet.

Our compiler is whole program, although we could split it up by defining interface points which are not violated. We haven't bothered to do so yet.

\section{Case Studies}

We have aimed for a practical supercompiler, and in this section we outline two practical purposes to which we have already deployed our supercompiler.

\subsection{Optimisation of HTML Parsing}
\label{sec:tagsoup}

The TagSoup library \cite{tagsoup} is a simple parser for XML/HTML, based on the HTML 5 specification. Given a String, TagSoup produces a list of tokens (such as tag open, tag close, attribute). The parser was deliberately written in a way that mirrors the HTML 5 specification, which is based around a state passing approach. Each rule has been modelled in the most direct way, and then a supporting library simplifies it. For example, section 9.2.4.10 of the HTML 5 specification states:

\begin{quote}
9.2.4.10 Attribute value (double-quoted) state

Consume the next input character:

U+0022 QUOTATION MARK (") - Switch to the after attribute value (quoted) state.

U+0026 AMPERSAND (\&) - Switch to the character reference in attribute value state, with the additional allowed character being U+0022 QUOTATION MARK (").

EOF - Parse error. Reconsume the EOF character in the data state.

Anything else - Append the current input character to the current attribute's value. Stay in the attribute value (double-quoted) state.
\end{quote}

And the corresponding code is:

\begin{code}
-- 9.2.4.10 Attribute value (double-quoted) state
attValueDQuoted S{..} = pos $ case hd of
    '\"' -> afterAttValueQuoted tl
    '&' -> charRefAttValue attValueDQuoted (Just '\"') tl
    _ | eof -> errWant "\"" & dat s
    _ -> hd & attValueDQuoted xml tl
\end{code}

Here |tl| is the next state, |hd| is the current character, and the initial |pos| call emits position information. The |(&)| operator is used to place a token on the output stream. However, this high level of abstraction has a noticeable performance penalty, for each output token there are several intermediate values created. While work on list fusion can often reduce intermediate lists, the values here have much more structure than lists, and thus this work is not appropriate. There is a strong desire not to complicate the specification by adding details that improve performance.

The code is split as 308 lines translated from the spec, followed by 191 lines implementing the operations and putting together the results in the right format.

We slightly prime the supercompiler. The optimisation is controlled by 4 booleans, and by freezing them we manage to take fast paths -- for example when generating a stream without position information then |pos| calls are entirely eliminated. We did the same trick with GHC, but it was negligible -- mainly we suspect that the SpecConstr wasn't able to specialise through all the loops, which would have eliminated it. Plus there is no opportunity for fusion.

One might ask whether optimising an HTML parser is worthwhile, but the answer is decidedly yes. The TagSoup library is used for DNA processing, and nightly gets run over 40Gb of XML files. The bottleneck is currently TagSoup, but with these transformations we eliminate that.

Note that our HTML parser is a perfect use case for supercompilation. There is no hot-spot in the program that takes up more time, the problem is that the overhead of the abstraction is throughout. The abstraction is nicely chosen to map to an external document, so the abstraction cannot easily be altered. We are also constrained by speed. Supercompilation delivers nicely, removing all the abstractions to make them valuable at compile time, and yet removed by runtime.

\subsection{Equality of Open Expressions}
\label{sec:hlint}

The HLint program \cite{hlint} is a tool for helping improve Haskell source code. A large number of its hints are based on replacing one open expression with another. For example, if the user writes |concat (map f (xs++ys))| it will suggest replacing it with |concatMap f (xs++ys)|. It works by having a list of open expressions it uses for replacement:

\begin{code}
forall f x . concat (map f x) ==> concatMap f x
\end{code}

These rules are written in a supporting file, and there are many of them. Recently one bug was filed stating that one of the rules (involving |foldl| and |map| fusion) was incorrect. After fixing that another incorrect hint was discovered. The intention is clearly that in most cases the left and right expressions are equal. More accurately, the left and right open expressions should be equal. A supercompiler transforms a program and often produces equivalent expressions.

We supercompiled the left and right hand sides. Of the \unknown{} rules, they fit in to three categories:

\paragraph{Type class based on both sides}

If both the left and the right have type classes, i.e. |not (a == b) ==> a /= b|, we can't check anything. This rule is not true in Haskell unless the standard typeclass rules have been followed. This pattern accounts for \unknown{} rules.

\paragraph{Generalisation}

For example |(\(x,y) -> (f x, g y)) ==> f *** g|, while this is somewhat true, it is actually a generalisation. (Note that there is a second condition that f and g do not contain x or y in them). The first expression works on tuples only, while the second is generalise to all arrrows -- of which tuples are one special case. This pattern accounts for 2 rules.

\paragraph{Remaining examples}

These are examples where the equality of both sides is true, and both should be equivalent. Of the remaining ones, we can prove them. Most examples are trivial, but some are slightly more involved (i.e. map fusion rules). For example we were able to prove:

\begin{code}
\end{code}

We can't prove \unknown{} examples. Here are two representative examples:

\paragraph{Generalisation}

\begin{code}
(if a then True else False) ==> a
\end{code}

Here the examples are not equivalent, technically, as the right could be of any type. It would be possible to compress the right hand side in a post processing, but we don't yet do that.

\paragraph{Strictness}

We cannot prove:

\begin{code}
error "Use isPrefixOf" = (take i s == t) ==> ((i == length t) && (t `isPrefixOf` s))
\end{code}

The idea here is that people write |take 4 s == "ICFP"|, when they should have written |"ICFP" `isPrefixOf` s|. Usually the first term |(4 == length "ICFP")| will be eliminated. In addition the rule has a side condition that both |i| and |t| must be concrete literals. The supercompiler produces different expressions for both sides. this is in fact important -- the first expression is lazy in the spine of the |t|, while the right hand side is not. Unless the condition is applied, which the supercompiler can't see.

We currently have no way to tell the supercompiler that an expression is fully evaluated, but perhaps we should and include strictness information throughout.

\paragraph{Finding a bug}

We found a bug. In HLint 1.6.19 there was a rule to reduce |foldr/map| with the rule:

\begin{code}
foldr f z (map g x) ==> foldr (f . g) z x
\end{code}

Unfortunately we also got the equivalent for |foldr1|, namely:

\begin{code}
foldr1 f (map g x) ==> foldr1 (f . g) x
\end{code}

This rule is not true, and our supercompiler spotted it. \footnote{Note that in this case the types are different, which should have given the game away. Unfortunately our type check method simply checks the types can unify (a very cheap check) which doesn't pick this up.}

\section{Related Work}

We have extensively covered the related work in supercompilation in \S\ref{sec:comparison}. Our work is definately classed as a supercompiler, but makes a large number of novel decisions. In particular, where often supercompiler authors have a choice at some points, many of our design decisions are forced upon us. We hope that this leads to decisions that naturally fit together. Whatever the result, our supercompiler is certainly a new point on the design space.

Work on list fusion etc. is usually limited to lists, which isn't great.

Partial evaluation has it's place, but supercompilers tend to be better for changing data -- for example the tagsoup would have frozen the parameters with partial evaluation, but not eliminated intermediate data structures.

One intruiging possibility is that our supercompiler as described may actually be a great target for partial evaluation. The program being supercompiled is static, and in our supercompiler is not perturbed to the extent of other supercompilers.

Many program optimisation techniques reduce abstraction, some such as fusion and specialisation have been covered extensively. However, the neat thing about supercompilation is that fusion (particularly list fusion) falls out naturally by the way the program is written -- no specific rules such as foldr/build or stream/unstream. In particular, we can fuse away |words . length|, even though in general there is no fusion rule to do that.



\section{Conclusions and Future Work}

Need more benchmarks, integrate into a production quality compiler (GHC). Our supercompiler has been designed to run faster, but does it really do so?

\subsection{Compile Time Performance}

We can implement this construct much more efficiently, in particular significantly more efficiently than a homeomorphic embedding. We can reify this sequence as |Map Name Int|.

We can change names to be a function name and subexpression index combined. We can also free up the last 6 bits to store the constructor count, allowing over 60 million subexpressions even with 32 bit integers. We now have very efficient names.

We can pre-resolve in to a decision table or finite state machine very easily. However, it's not really necessary - the basic test is dead fast, if you use ordered lists.

We currently keep a list of history built using |(:)|, but that's very inefficient. If instead we have a termination context that can manipulate expressions and store more structure it can go faster.

\subsection{Run Time Performance}

What about merging a rules engine. What about strictness analysis.

\subsection{Conclusions}

We have presented a supercompiler which is simple, and has found practical use. What a great result.


\subsection{Acknowledgements}

Thanks to lots of people for helpful ideas and discussions. Particularly I'd like to thank Jason Reich, Simon Peyton Jones, Max Bolingbroke and Peter Jonsson.

\bibliographystyle{plainnat}
\bibliography

\end{document}
