\documentclass{sigplanconf}

\usepackage{amsmath}
\usepackage{amssymb}
\usepackage{natbib}
\usepackage{multirow}
\usepackage{setspace}
\usepackage{balance}

\include{paper}
%include paper.fmt

%format << = "[\!["
%format >> = "]\!]"
%format += = "+\!\!\!\!="

\newcommand{\unknown}{XXX}

\begin{document}

\conferenceinfo{ICFP 2010}{}
% \CopyrightYear{2009}
% \copyrightdata{978-1-60558-508-6/09/09}

\titlebanner{\today{} - \currenttime{}}        % These are ignored unless
\preprintfooter{}   % 'preprint' option specified.

\title{Simpler Supercompilation}
% \subtitle{}

\authorinfo{Neil Mitchell}
           {Standard Chartered, UK}
           {\verb"ndmitchell@gmail.com"}

\maketitle

\begin{abstract}
Supercompilation is a technique for program optimisation, which can often eliminate the overhead of programmer abstractions. We present a new design for a supercompiler, rethinking many of the decisions that have become common in previous supercompilation work. The result is a supercompiler that we find simpler, and offers good performance both at compile time and runtime. We have implemented our supercompiler and benchmarked it a selection of programs from the nofib benchmark suite, which run faster than both GHC and our previous supercompiler. We then put our supercompiler to practical use, optimising an HTML parsing library, and checking the equivalence of open expressions used as rewrite rules.
\end{abstract}

\category{D.3}{Software}{Programming Languages}

\terms
Languages

\keywords
Haskell, optimisation, supercompilation

\section{Introduction}

\todo{perhaps switch to tagsoup as the openning example, so I can come back to it?}

Supercompilation is a technique for program optimisation, that is particularly suited to removing the overhead introduced by abstractions. Consider a program that counts the number of words read from the standard input -- in Haskell \cite{haskell} this can be compactly written as:

\begin{code}
main = print . length . words =<< getContents
\end{code}

Reading the program right to left, we first read the standard input as a string (|getContents|), then split it in to words (|words|), count the number of words (|length|), and print the result (|print|). An equivalent C program is unlikely to use such a high degree of abstraction, and is more likely to get characters and operate on them in a loop using some state which is updated.

Sadly the C program is three times faster, even using the advanced optimising compiler GHC \cite{GHC}. The abstractions that make the program concise have a significant runtime cost. In a previous paper on supercompilation \cite{me:supero} we showed how supercompilation can remove these abstractions, to the stage where the Haskell is even faster than the C version (by about 6\%). In the Haskell program after optimisation all the intermediate lists have been removed, and the |length . words| part of the pipeline is translated into a state machine.

One informal description of supercompilation is that you simply ``run the program at compile time''. This description leads to two questions -- what happens if you are blocked on information only available at runtime, and how do you ensure termination? Answering these questions provides the design for a supercompiler. This paper strives to provide simple answers, but which do not reduce the optimisation power. In doing so, we make a large number of departures from the current concensus on supercompiler design.

We hope that from the descriptions given in this paper a user is able to write their own supercompiler. Indeed, we have written a supercompiler following this design which is available online \unknown{}.

\subsection{Contributions}

Our primary contribution is the design of a new supercompiler (\S\ref{sec:method}). Our supercompiler has many differences from previous supercompilers (\S\ref{sec:comparison}), including a new core language, a substantially different treatment of let expressions, a transformation that never manipulates inside expressions and an entirely new termination criteria. The result is relatively simple, yet still powerful.

In addition, we make the following contributions:

\begin{itemize}
\item We give examples of how our supercompiler performs (\S\ref{sec:examples}), including how it subsumes list fusion and specialisation, and what happens when the termination criteria are needed.
\item We benchmark the supercompiler on a small range of examples (\S\ref{sec:benchmarks}) -- our supercompiler achieves an improvement of \unknown{}\% when used in combination with GHC, as opposed to GHC alone.
\item We use our supercompiler to optimise an HTML parser, showing a \unknown{}\% speed increase (\S\ref{sec:tagsoup}). This HTML parser is run daily on 40Gb of input, and previously consumed over two hours of CPU time per day.
\item We use our supercompiler to prove the equality of open expressions for a program rewriter (\S\ref{sec:hlint}). As a result we found one bug.
\end{itemize}

\section{Method}
\label{sec:method}

This section describes our design for a supercompiler. We first present a Core language (\S\ref{sec:core}), along with rules for normalisation (\S\ref{sec:simplify}). We then present the overall algorithm in (\S\ref{sec:manager}, which ties together the answers to the questions:

\begin{itemize}
\item How do you evaluate an open term? (\S\ref{sec:eval})
\item What happens if you can't evaluate an open term any further? (\S\ref{sec:eval_split})
\item How do you know when to stop? (\S\ref{sec:term})
\item What happens if you have to stop? (\S\ref{sec:term_split})
\end{itemize}

Throughout we make use of the following example:

\begin{code}
root f g x = map g (map f x)

map f x = case x of
    []    -> []
    y:ys  -> f y : map f ys
\end{code}

This example applies map twice -- the expression |map f x| produces a list that is immediately consumed by the outer |map|. List fusion \cite{gill:shortcut_deforestation} can be used to remove the intermediate list, and a good supercompiler should also remove the intermediate list.

\subsection{Core Language}
\label{sec:core}

\begin{figure}
\begin{code}
type Var  = String
type Fun  = String
type Con  = String

data Exp  = Var Var
          | Fun Fun
          | Con Con [Var]
          | App Var Var
          | Let [(Var,Exp)] Var
          | Case Var [(Pat, Exp)]

type Pat = Exp -- restricted to Con
\end{code}
\caption{Core Language}
\label{fig:core}
\end{figure}

Our Core language for expressions is given in Figure \ref{fig:core}.

Our Core language is similar to many others. A few points to note:

\begin{itemize}
\item We have Var in many places that would normally have an expression. A standard Core language could be translated to ours by inserting let bindings.
\item In typical Core languages a distinction is made between let and letrec. Our let binding does allow values bound in a variable block to be used within that let block, but there must be an equivalent rewrite which doesn't use letrecs. This flexibility comes in handy later on.
\item We don't have default patterns in case expressions, or literals. Both can be added (see \S\ref{sec:extensions}), but are of little interest while working with the supercompiler.
\end{itemize}

We write expressions using standard Haskell syntax (i.e. |let| for |Let|, |case/of| for |Case| etc.). Rewriting our example in this reduced Core language gives:

\begin{code}
root f g x =
    let v_1 = map
        v_2 = f
        v_3 = let w_1 = map
                  w_2 = g
                  w_3 = x
                  w_12 = w_1 w_2
                  w_123 = w_12 w_3
              in w_123
        v_12 = v_1 v_2
        v_123 = v_12 v_3
    in v_123

map f x = case x of
    [] -> []
    y:ys -> let v_1 = f
                v_2 = y
                v_12 = v_1 v_2
                w_1 = map
                w_2 = f
                w_3 = ys
                w_12 = w_1 w_2
                w_123 = w_12 w_3
                r = (:) v_12 w_123
            in r
\end{code}

Of course, this notation is rather tedious. Instead of often use |<<| brackets |>>| to denote that we later replace them, and typically only expand the root, leaving |map| as:

\begin{code}
map f x = case x of
    [] -> []
    y:ys -> << f y : map f ys >>
\end{code}


\subsection{Normalised Core}
\label{sec:simplify}

Given our Core language above it is easy to define a normalised simplified form. Our simplified form requires that the root of a function be a |Let|, and that the bindings of that |Let| must not be of the following form:

\begin{itemize}
\item |App v w|, where |v| is bound to a |Con| - the |App| can be replaced with a Con of higher arity.
\item |Case v w|, where |v| is bound to a |Con| - the expression can be explored.
\item |Let vs w|, where any of the expressions bound in |vs| is a |Let| - the expression can be lifted - renaming if necessary.
\item |Let vs w|, where any of the expressions bound in |vs| is a |Var| - we can replace the expression with it's inlining.
\item All bound variables are unique.
\end{itemize}

For example with normalisation we write |root| as:

\begin{code}
root f g x =
    let v_1 = map
        w_1 = map
        w_12 = w_1 g
        w_123 = w_12 x
        v_12 = v_1 f
        v_123 = v_12 w_123
    in v_123
\end{code}

We can also normalise the names in the expression trivially, but tend not when presenting examples to make them easier to follow. We could mandate that common subexpression elimination was applied, but have not done so.

Note that we do not enforce details about nested expressions, for example the following is legal:

\begin{code}
f x =
    let v = case x of
                Pat -> let w = True ; q = w in q
    in v
\end{code}

The binding |q = w| would be disallowed in the outermost let, but not in the inner ones. It is important to note that normalisation never explores the depths of an expression. This property is important for the way we prove termination.

One possible extension would be to automatically expand some functions whose termination was guaranteed - for example |($)|, |(.)|, |const|, |id|, |otherwise|. This could only be done for expressions which decreased in size.

\subsection{Manager}
\label{sec:manager}

Many supercompilers are based around the concept of ``driving'' expressions, where the code traverses an expression applying optimisation. We instead prefer to split the evaluation (or driving) of expressions from the other aspects of a supercompiler. The idea of the manager is that it makes use of the other parts of a supercompiler.

Our method can be summarised in three points:

\begin{itemize}
\item Given an open expression, we evaluate it for a number of steps, then split it into a residual function (which goes in the result program), and a list of inner expressions which we supercompile.
\item If we ever encounter an expression we've already supercompiled, we use the residual function from before.
\item When evaluating an open expression we stop when either we can't evaluate any further (a free variable is in the way), or when a termination criteria says we should.
\end{itemize}

This method is encapsulated in our |supercompile| function in Figure \ref{fig:supercompile}, and the helper functions in Figure \ref{fig:result}. The |Result| functions take care of spotting seen functions and inserting forwarding expressions etc. Broadly speaking, point 1 is addressed by |supercompile|, point 2 is addressed by the |Result| monad and point 3 is addressed by |reduce| with a little help from |supercompile|.

Termination is handled by having a context of which expressions have been seen before (encapsulated by |Term|), which is added to with the |(+=)| operator, and queried with the |termination| function (\S\ref{sec:term}). Once we have decided to terminate, we call |stop| to reduce the expression (\S\ref{sec:term_split}).

We use the function |reduce| to evaluate an expression until it's stopped either by termination, or by not being able to evaluate it any further. The |reduce| function makes use of a local termination context.

\begin{figure}
\begin{code}
type Env = Fun -> Func
type Func = (Fun, [Var], Exp)
type FuncBody = ([Var], Exp)
type Split = ([FuncBody], [Fun] -> Func)

supercompile :: Env -> Term -> Func -> Result ()
supercompile env t = do
    let (xs,gen) = if terminate t x then stop t x else reduce env x
    ys <- map assignName xs
    addResult $ gen $ map funcName ys
    mapM_ (supercompile env (t+=x) ys
\end{code}
\caption{The |supercompile| function.}
\label{fig:supercompile}
\end{figure}

\begin{figure}
\begin{code}
data R = R {seen :: [(FuncBody,Fun)], result :: [Func], fresh :: [Fun]}
type Result alpha = State R alpha

-- We use the syntactic sugar
-- |label := value = modify $ \r -> r{label = value}|

assignName :: FuncBody -> Result Func
assignName body = do
    seen <- gets seen
    case lookup body seen of
        Just y -> return $ func y body
        Nothing -> do
            name:fresh <- gets fresh
            seen := (bod,name) : seen
            return $ func name body

addResult :: Func -> Result ()
addResult func = do
    seen <- gets seen
    case lookup (funcBody func) seen of
        Just y -> do
            seen := [(a, if b == funcName func then y else b) | (a,b) <- seen]
            result := forward (funcName func) y : result
        Nothing -> do
            seen := (funcBody func, funcName func) : seen
            result := func
\end{code}
\caption{The |Result| monad.}
\label{fig:result}
\end{figure}


The important point about the driver is that we terminate, we use a fresh driving expression per thing, and we get reduced forms out.

So, harping back to our initial example, we first evaluate until we get:

\begin{code}
root f g x = << map f (map g x) >>
\end{code}


\begin{code}
case x of
   [] -> []
   x:xs -> f (g x) : ...
\end{code}

We then end up tying back so in the |...| we write |root f g xs|, making a fused version.

We have presented our manager in a great amount of detail -- deliberately. When writing a supercompile it turns out that the most subtle aspect is actually the manager. There are some important points -- you must always assign names before making the supercompile call (to ensure that recursive functions terminate). The use of forwarding methods is important to keep the complexity down, and can be resolved during post processing (\S\ref{sec:postprocess}). Very small changes result in it not terminating. The only differences between this code, and the code used in our actual supercompiler, is that our version has additional logging behaviour to allow the effects of supercompilation to be debugged!

\subsection{Evaluation}
\label{sec:eval}

The evaluation step is easy, follow from the root expression upwards until we hit a |Fun|. Once we do, expand it out replacing with it's body, provided it has enough arguments not to be a lambda.

\subsubsection{One Step Evaluation}

We can transform our code in to a zipper:

\begin{code}
force :: Exp -> Maybe Var
force (Case v _) = Just v
force (App v _) = Just v
force (Var v) = Just v
force _ = Nothing

stack :: Func -> [(Var, Exp)]
stack (Func _ free (Let bind v)) = f v
    where f v = case lookup v bind of
                     Nothing -> []
                     Just x -> maybe [] f (force x) ++ [(v,x)]
\end{code}

i.e. for the root expression, we'd end up with the stack:

\begin{code}
stack here
\end{code}

Now we can define evaluation as:

If the stack matches:

\begin{code}
stack[0] = (v_1, Fun x)
Func _ args bod = env x
n = length args
stack[i] = (v_i, App _ a_i)   where i `elem` [1..n]
new = Let (zip args (map Var [a_1..a_n])) bod
\end{code}

Then we replace variable |v_n| with |new| in the original expression.

\subsubsection{Multiple Step Reduction}

And reduction is just:

\begin{code}
reduce :: Env -> Func -> Split
reduce env = f newTerm
    where f t x | terminate t x = stop t x
                | Just x' <- step env x = f (t += x) x'
                | otherwise = split x
\end{code}

How awesome.

Going back to our root evaluation, we'd end up with the sequence...

\subsection{Evaluation Splitting}
\label{sec:eval_split}

If evaluation cannot proceed any further then we look at which case we have arrived in, and make an appropriate split. The idea of the split is that we take whatever construct is blocking evaluation and solve it separately, leaving remaining terms which we can evaluate. There are three distinct cases.

\subsubsection{Case on a free variable}

If the top of the stack is a case on a free variable, then we residuate a case statement on the variable, and rerun the function injecting a binding for each possible pattern. Essentially we use residuation to gather knowledge about the variable then continue with the variable filled in. The normalisation will instantly cause the case to be reduced.

More concretely, if we have the pattern:

\begin{code}
stack[0] = (_, Case v alts)
\end{code}

We can transform to a residual expression:

\begin{code}
let a_i = supercompile $ whole with v set to reverse of p_i
in case v of zip p_is a_is
\end{code}


\subsubsection{All but under application}

This includes root being a constructor, a single variable, and an application to an unknown constructor. Note that if the top of the stack is a constructor of variable, then the stack must have exactly one item on it.

We do roughly the same trick, lifting the free variable upwards and residuating. In this case it's important that we don't loose sharing of any of the intermediate forms, so we first lift all the necessary ones upwards.

Step 1, we create a residual function where all variables are stored on let's. This is correct, but isn't very efficient. Given a set of variables |vs| binding to expressions |xs|, we can remove any binding |vi| if |vi| is only used within on expression (say |xj|, bound to |vj|), and we rewrite |xj| to be |let vi = xi in xj|. We can repeat this rule as often as we like to increase sharing.

Given the expression:

\begin{code}
stack[0] = (var, bod)
\end{code}

We residuate this to the full residual, but where we require that the root of the func (under the let), |var| and all the variables contained within |bod| all must be within the remaining bindings.

For example:

\begin{code}
let v = foo
    w = f v
    f = bar
    r = Con v w
in r
\end{code}

This would rewrite as:

\begin{code}
let v = << foo >>
    w = << let f = bar in f v >>
    r = Con v w
in r
\end{code}

Note that the |f| binding has been pushed under |w|, but no loss of sharing has occurred.

\subsubsection{Under Application}

Unfortunately, under application is hard. To deal with under application, we have to introduce a lambda. But we can't introduce a lambda without first binding all the free variables, as otherwise they'll get shared in bad ways.

We therefore write:

\begin{code}
let vs  = ...
    r  = \x -> ...
in r
\end{code}

We then move all variables from |v| under |r|, and crucially past the lambda, if |vs| is determined to be "simple". Currently we consider constructors to be simple. We don't like this heuristic as something that trivially evaluates to a constructor isn't considered simple.

We note that this isn't a problem in practice as supercompilation usually means a function will be saturated when it is being evaluated.


If we have not got enough arguments then we add some. In all cases this must be the root thing (as we can't case on a lambda, and a function requiring further arguments must be a lambda). We do however need to be careful about sharing, as there may be a binding outside a lambda which cannot be pushed inwards.

\begin{code}
let [bind]
in \v -> root v
\end{code}


\subsection{Termination}
\label{sec:term}

We tag each expression within the first let with it's origin. We then compare to previous values. The idea is that it must go down.

\subsubsection{Termination Rule}

In general, given a quasi normal rule:

\[
\forall i, j \bullet i < j \Rightarrow x_i \succ y_i
\]

which is awesome

\[
x \succ y = x \subset_{set} y \vee x \supset_{bag} y
\]

Given a bag of values drawn from the alphabet $\Sigma$, a sequence is well formed if there does not exist an $x_i \subset x_j, where j > i$. To give some examples:

Well formed:

[a,aaaaab,aaab,b]

If we just had the $\supset_{bag}$ then it would be obviously decreasing. We also allow $\subset_{set}$ but that can only occur a finite number of times as there are only a finite number of sets given a finite alphabet.

The termination is obvious because it must be. We have a finite alphabet, and an initial expression.

Given an alphabet $\Sigma$, each variable has been either seen or unseen.

If all variables have been seen, then you can prove there are a finite remaining number. If one variable hasn't been seen then either you never introduce it (in which case you).

Induction on the size of $\Sigma$:

$\Sigma = 1$ - you can have exactly one item, the second one terminates.

$\Sigma = n+1$. You have a sequence before you hit a sigma, if you were to never hit a sigma that would terminate. Then you hit a sigma with a finite value |m|.

\subsubsection{Tracking Names}

Every expression throughout the program is assigned a name, which is carefully tracked. A name is a triple, <function name, expression index, constructor count>. Within a function, every expression is assigned the name with the name of the function, and a uniquely chosen expression index for that function, and the constructor count 0. When manipulating expressions we only do one of two things: 1) extract a subexpression (which already has a name); 2) if the subexpression is a constructor we may add a number of variables to the end of it; 3) inject new constructors for evaluation splitting on case. The only case where we need to manufacture a new name is when adding to a constructor, and we do this by bumping the constructor count once for each variable. The one subtlty in this process is splitting on a case, which "invents" and injects a new constructor binding. In this case we simply assign each pattern a name as though it were an expression.

It is easy to see there are a finite number of names. Each function labels it's subexpressions in advance, so there can only be a finite number. We only increase the counstructor count to get new names, and we can only do this up to the arity of a given constructor. Therefore, there are only a finite number of names in circulation.

We could merge the function name and subexpression index, but that would be harder to track down which expressions come from where.

\subsubsection{Example}

Many small examples programs do not encounter the termination criteria, certainly the map/map from the start does not. In fact, to hit them the program usually has to store a buffer of memory which it uses later. The classic example of this pattern is |reverse|, which builds up an accumulated list which it uses later.

\begin{code}
reverse xs = foldl (flip (:)) [] xs
\end{code}

Here the reverse call builds up arguments, which are limited only by the size of |xs|. We need to not buffer these arguments, or we come undone.

We follow it through, and show that it requires termination at the right place. Note that we have unfolded exactly one buffering of the accumulator, which is rather nice.

\begin{code}
! <<reverse xs>>
\xs -> foldl snoc [] xs

<<foldl snoc [] xs>>
\xs -> case xs of
    [] -> []
    x:xs -> foldl snoc (snoc [] x) xs

! <<foldl snoc (snoc [] x) xs>>
\x xs -> case xs of
    [] -> x:[]
    y:ys -> foldl snoc (snoc (snoc [] x) y) ys

! <<foldl snoc (snoc (snoc [] x) y) ys>>
! <<foldl snoc (snoc t y) ys>>

! <<foldl snoc (snoc t y) xs>>
\x xs -> case xs of
    [] -> x:[]
    z:zs -> foldl snoc (snoc (snoc t y) z) zs

! <<foldl snoc (snoc (snoc t y) z) zs>>
! <<foldl snoc (snoc t y) xs>>
\end{code}


\subsubsection{Argument for Termination}

We use |(+=)| to add to the sequence. We use |terminate| to check that nothing violates the sequence.

\subsubsection{Efficient Implementation}

We can implement this construct much more efficiently, in particular significantly more efficiently than a homeomorphic embedding. We can reify this sequence as |Map Name Int|.

We can change names to be a function name and subexpression index combined. We can also free up the last 6 bits to store the constructor count, allowing over 60 million subexpressions even with 32 bit integers. We now have very efficient names.

We can pre-resolve in to a decision table or finite state machine very easily. However, it's not really necessary - the basic test is dead fast, if you use ordered lists.

\subsection{Termination Splitting}
\label{sec:term_split}

We don't need to ensure anything too bad, just that at the end we no longer hit the termination criteria. If we were to bind all variables separately at the top-level, then we'd no longer have termination (by the rule of singleton sets never being too small).

We start with all variables at the top, then increase sharing as in \S\ref{sec:eval_split}, with the rule that none of the termination criteria are violated. Running over the map/map fusion example from above we never hit the termination criteria.

\subsection{Argument for Correctness}

We have not shown the correctness of our supercompiler, but have tested it extensively. The termination argument follows from each section separately, in that we only ever transform an expression in semantics preserving ways.

\subsection{Post-processing}

We often need to post-process quite a bit, in particular we make the following simple transformations.

\subsection{Comparison to Other Supercompilers}
\label{sec:comparison}

It has no lambda, which makes it different from most, but exactly like Jason Reich's.

It has let everywhere, unlike most supercompilers which almost relegate let to a second class system.

It doesn't use homeomorphic embedding.

It never rewrites expressions, allowing for the tantilising opportunity to partially evaluate the supercompiler.

We use a local termination pile, which means the optimisation of a program is not based on what has passed before. This hopefully leads to better predictability.

It's simple. The details provided in this paper are complete -- we have left nothing out.

\subsection{Extensions}
\label{sec:extensions}

A real supercompiler needs slightly more flexibility, and the one we have implemented does indeed have all these

\subsubsection{Primitives}

Primitive functions are crucially important, but can be handled by simply treating them like free variables - everything works out.

\subsubsection{Literals}

We add primitives in to the language, and allow |Patt| to be both |Con| and |Lit|. In all other respects we can ignore primitives.

\subsubsection{Case Defaults}

Again, we just augment Patt. There are slight complexities to evaluation splitting when you are dealing with the default branch (don't bind the variable, just project out the right branch), but these aren't too challenging.

\section{Examples}
\label{sec:examples}

Here are some examples of supercompiled functions, with a little bit of commentary.

\subsection{Specialisation}

Give map head as an example.

\subsection{Fusion}

You've already seen map/map, but it's important to remember that with our supercompiler lists are not special in any way. For example, if we wrote the identical program on the data type |data MyList alpha = MyNil || MyCons alpha (MyList alpha)| we'd get exactly the same fusion. To give an example:

\begin{code}
data Tree alpha = Leaf alpha | Branch (Tree alpha) (Tree alpha)

root x = flatten (mapTree (+1) x)
flatten -- with continuations
mapTree -- standard fmap on Tree
\end{code}

And we get the right result out.

\subsection{Residual values}

Of course, we can't specialise out |reverse|, for example:

\begin{code}
reverse = foldl (flip (:)) []
\end{code}

But we do get a nice efficient version of |reverse|, that has |foldl| specialised and optimised.

\section{Benchmarks}
\label{sec:benchmarks}

The standard nofib ones, eek!

\subsection{Execution Speed}

See how great we do, yay.

\subsection{Compilation Speed}

Our compiler is quite quick, we've broken in down in to the time to compile and the result. The tricks we use to speed up compilation are variable normalising, using a map for let expressions, and we could also reduce the termination keys in to |Int|, but we haven't bothered yet.

Our compiler is whole program, although we could split it up by defining interface points which are not violated. We haven't bothered to do so yet.

\section{Case Studies}

We have aimed for a practical supercompiler, and in this section we outline two practical purposes to which we have already deployed our supercompiler.

\subsection{Optimisation of HTML Parsing}
\label{sec:tagsoup}

The TagSoup library \cite{tagsoup} is a simple parser for XML/HTML, based on the HTML 5 specification. Given a String, TagSoup produces a list of tokens (such as tag open, tag close, attribute). The parser was deliberately written in a way that mirrors the HTML 5 specification, which is based around a state passing approach. Each rule has been modelled in the most direct way, and then a supporting library simplifies it. For example, section 9.2.4.10 of the HTML 5 specification states:

\begin{quote}
9.2.4.10 Attribute value (double-quoted) state

Consume the next input character:

U+0022 QUOTATION MARK (") - Switch to the after attribute value (quoted) state.

U+0026 AMPERSAND (\&) - Switch to the character reference in attribute value state, with the additional allowed character being U+0022 QUOTATION MARK (").

EOF - Parse error. Reconsume the EOF character in the data state.

Anything else - Append the current input character to the current attribute's value. Stay in the attribute value (double-quoted) state.
\end{quote}

And the corresponding code is:

\begin{code}
-- 9.2.4.10 Attribute value (double-quoted) state
attValueDQuoted S{..} = pos $ case hd of
    '\"' -> afterAttValueQuoted tl
    '&' -> charRefAttValue attValueDQuoted (Just '\"') tl
    _ | eof -> errWant "\"" & dat s
    _ -> hd & attValueDQuoted xml tl
\end{code}

Here |tl| is the next state, |hd| is the current character, and the initial |pos| call emits position information. The |(&)| operator is used to place a token on the output stream. However, this high level of abstraction has a noticeable performance penalty, for each output token there are several intermediate values created. While work on list fusion can often reduce intermediate lists, the values here have much more structure than lists, and thus this work is not appropriate. There is a strong desire not to complicate the specification by adding details that improve performance.

The code is split as 308 lines translated from the spec, followed by 191 lines implementing the operations and putting together the results in the right format.

We slightly prime the supercompiler. The optimisation is controlled by 4 booleans, and by freezing them we manage to take fast paths -- for example when generating a stream without position information then |pos| calls are entirely eliminated. We did the same trick with GHC, but it was negligible -- mainly we suspect that the SpecConstr wasn't able to specialise through all the loops, which would have eliminated it. Plus there is no opportunity for fusion.

One might ask whether optimising an HTML parser is worthwhile, but the answer is decidedly yes. The TagSoup library is used for DNA processing, and nightly gets run over 40Gb of XML files. The bottleneck is currently TagSoup, but with these transformations we eliminate that.

Note that our HTML parser is a perfect use case for supercompilation. There is no hot-spot in the program that takes up more time, the problem is that the overhead of the abstraction is throughout. The abstraction is nicely chosen to map to an external document, so the abstraction cannot easily be altered. We are also constrained by speed. Supercompilation delivers nicely, removing all the abstractions to make them valuable at compile time, and yet removed by runtime.

\subsection{Equality of Expressions}
\label{sec:hlint}

The HLint program \cite{hlint} is a tool for helping improve Haskell source code. A large number of its hints are based on replacing one open expression with another. For example, if the user writes |concat (map f (xs++ys))| it will suggest replacing it with |concatMap f (xs++ys)|. It works by having a list of open expressions it uses for replacement:

\begin{code}
forall f x . concat (map f x) ==> concatMap f x
\end{code}

These rules are written in a supporting file, and there are many of them. Recently one bug was filed stating that one of the rules (involving |foldl| and |map| fusion) was incorrect. After fixing that another incorrect hint was discovered. The intention is clearly that in most cases the left and right expressions are equal. More accurately, the left and right open expressions should be equal. A supercompiler transforms a program and often produces equivalent expressions.

We supercompiled the left and right hand sides. Of the \unknown{} rules, they fit in to three categories:

\paragraph{Type class based on both sides}

If both the left and the right have type classes, i.e. |not (a == b) ==> a /= b|, we can't check anything. This rule is not true in Haskell unless the standard typeclass rules have been followed. This pattern accounts for \unknown{} rules.

\paragraph{Type class on the right hand side}

For example |(\(x,y) -> (f x, g y)) ==> f *** g|, while this is somewhat true, it is actually a generalisation. (Note that there is a second condition that f and g do not contain x or y in them). The first expression works on tuples only, while the second is generalise to all arrrows -- of which tuples are one special case. This pattern accounts for 2 rules.

\paragraph{Remaining examples}

These are examples where the equality of both sides is true, and both should be equivalent. Of the remaining ones, we can prove them. Most examples are trivial, but some are slightly more involved (i.e. map fusion rules). For example we were able to prove:

\begin{code}
\end{code}

We can't prove \unknown{} examples. Here are two representative examples:

\paragraph{Reduction}

\begin{code}
(if a then True else False) ==> a
\end{code}

Here the examples are not equivalent, technically, as the right could be of any type. It would be possible to compress the right hand side in a post processing, but we don't yet do that.

\paragraph{Strictness}

We cannot prove:

\begin{code}
error "Use isPrefixOf" = (take i s == t) ==> ((i == length t) && (t `isPrefixOf` s))
\end{code}

The idea here is that people write |take 4 s == "ICFP"|, when they should have written |"ICFP" `isPrefixOf` s|. Usually the first term |(4 == length "ICFP")| will be eliminated. In addition the rule has a side condition that both |i| and |t| must be concrete literals. The supercompiler produces different expressions for both sides. this is in fact important -- the first expression is lazy in the spine of the |t|, while the right hand side is not. Unless the condition is applied, which the supercompiler can't see.

We currently have no way to tell the supercompiler that an expression is fully evaluated, but perhaps we should and include strictness information throughout.

\paragraph{Finding a bug}

We found a bug. In HLint 1.6.19 there was a rule to reduce |foldr/map| with the rule:

\begin{code}
foldr f z (map g x) ==> foldr (f . g) z x
\end{code}

Unfortunately we also got the equivalent for |foldr1|, namely:

\begin{code}
foldr1 f (map g x) ==> foldr1 (f . g) x
\end{code}

This rule is not true, and our supercompiler spotted it. \footnote{Note that in this case the types are different, which should have given the game away. Unfortunately our type check method simply checks the types can unify (a very cheap check) which doesn't pick this up.}

\section{Related Work}

We have extensively covered the related work in supercompilation in \S\ref{sec:comparison}. Our work is definately classed as a supercompiler, but makes a large number of novel decisions. In particular, where often supercompiler authors have a choice at some points, many of our design decisions are forced upon us. We hope that this leads to decisions that naturally fit together. Whatever the result, our supercompiler is certainly a new point on the design space.

Work on list fusion etc. is usually limited to lists, which isn't great.

Partial evaluation has it's place, but supercompilers tend to be better for changing data -- for example the tagsoup would have frozen the parameters with partial evaluation, but not eliminated intermediate data structures.

One intruiging possibility is that our supercompiler as described may actually be a great target for partial evaluation. The program being supercompiled is static, and in our supercompiler is not perturbed to the extent of other supercompilers.

Many program optimisation techniques reduce abstraction, some such as fusion and specialisation have been covered extensively. However, the neat thing about supercompilation is that fusion (particularly list fusion) falls out naturally by the way the program is written -- no specific rules such as foldr/build or stream/unstream. In particular, we can fuse away |words . length|, even though in general there is no fusion rule to do that.



\section{Conclusions and Future Work}

Need more benchmarks, integrate into a production quality compiler (GHC). Our supercompiler has been designed to run faster, but does it really do so?

We have presented a supercompiler which is simple, and has found practical use. What a great result.


\subsection{Acknowledgements}

Thanks to lots of people for helpful ideas and discussions. Particularly I'd like to thank Jason Reich, Simon Peyton Jones, Max Bolingbroke and Peter Jonsson.

\bibliographystyle{plainnat}
\bibliography

\end{document}
