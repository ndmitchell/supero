\documentclass{llncs}

\begin{document}

\title{Fast Haskell}

\author{Neil Mitchell and Colin Runciman \inst{1}}

\institute{University of York, UK, \url{http://www.cs.york.ac.uk/~ndm}}

\maketitle              % typeset the title of the contribution

\begin{abstract}
This approach gives fast-ish compilation times and dead quick code.
\end{abstract}

\section{Introduction}

Haskell can be used to write high level specifications of programs, which can run. Take for example the task of counting the number of words in a file read from the input. In Haskell this program is a single line:

\begin{code}
main = print . length . words =<< getContents
\end{code}

Compared to a C program to perform this task (see Figure \ref{fig:c_words}), the Haskell version is clearly more concise, more specification orientated and clearly correct. Unfortunately with all these good points,  it is also three times slower than the C version.

There are several reasons that the Haskell version does not perform at the same speed as the C version:

\begin{description}
\item[Intermediate Lists] One of the reasons for the slowdown in the Haskell version is that many intermediate lists are produced and consumed as the program proceeds. The |getContents| function produces a list of characters, |words| consumes this list and produces a list of a list of characters, |length| then consumes the outermost list. The equivalent C program has no allocation.
\item[Higher Order Arguments] The |words| function is defined using the function |span|, which takes a predicate and splits the input list when the predicate becomes true. The predicate is passed as a higher order function, which is constant on all applications.
\item[Laziness] The Haskell program proceeds in a lazy manner, demanding one character from |getContents|, then processing it with each of the functions in the pipeline. At each stage, a lazy thunk for the remainder of the function is created.
\end{description}




\end{document}
